\documentclass[]{article}
\usepackage{lmodern}
\usepackage{amssymb,amsmath}
\usepackage{ifxetex,ifluatex}
\usepackage{fixltx2e} % provides \textsubscript
\ifnum 0\ifxetex 1\fi\ifluatex 1\fi=0 % if pdftex
  \usepackage[T1]{fontenc}
  \usepackage[utf8]{inputenc}
\else % if luatex or xelatex
  \ifxetex
    \usepackage{mathspec}
  \else
    \usepackage{fontspec}
  \fi
  \defaultfontfeatures{Ligatures=TeX,Scale=MatchLowercase}
\fi
% use upquote if available, for straight quotes in verbatim environments
\IfFileExists{upquote.sty}{\usepackage{upquote}}{}
% use microtype if available
\IfFileExists{microtype.sty}{%
\usepackage{microtype}
\UseMicrotypeSet[protrusion]{basicmath} % disable protrusion for tt fonts
}{}
\usepackage[margin=1in]{geometry}
\usepackage{hyperref}
\PassOptionsToPackage{usenames,dvipsnames}{color} % color is loaded by hyperref
\hypersetup{unicode=true,
            pdftitle={Using Prior Knowledge to Inform Future Restoration Activities in the Gulf of Mexico},
            colorlinks=true,
            linkcolor=Maroon,
            citecolor=Blue,
            urlcolor=blue,
            breaklinks=true}
\urlstyle{same}  % don't use monospace font for urls
\usepackage{graphicx,grffile}
\makeatletter
\def\maxwidth{\ifdim\Gin@nat@width>\linewidth\linewidth\else\Gin@nat@width\fi}
\def\maxheight{\ifdim\Gin@nat@height>\textheight\textheight\else\Gin@nat@height\fi}
\makeatother
% Scale images if necessary, so that they will not overflow the page
% margins by default, and it is still possible to overwrite the defaults
% using explicit options in \includegraphics[width, height, ...]{}
\setkeys{Gin}{width=\maxwidth,height=\maxheight,keepaspectratio}
\IfFileExists{parskip.sty}{%
\usepackage{parskip}
}{% else
\setlength{\parindent}{0pt}
\setlength{\parskip}{6pt plus 2pt minus 1pt}
}
\setlength{\emergencystretch}{3em}  % prevent overfull lines
\providecommand{\tightlist}{%
  \setlength{\itemsep}{0pt}\setlength{\parskip}{0pt}}
\setcounter{secnumdepth}{0}
% Redefines (sub)paragraphs to behave more like sections
\ifx\paragraph\undefined\else
\let\oldparagraph\paragraph
\renewcommand{\paragraph}[1]{\oldparagraph{#1}\mbox{}}
\fi
\ifx\subparagraph\undefined\else
\let\oldsubparagraph\subparagraph
\renewcommand{\subparagraph}[1]{\oldsubparagraph{#1}\mbox{}}
\fi

%%% Use protect on footnotes to avoid problems with footnotes in titles
\let\rmarkdownfootnote\footnote%
\def\footnote{\protect\rmarkdownfootnote}

%%% Change title format to be more compact
\usepackage{titling}

% Create subtitle command for use in maketitle
\newcommand{\subtitle}[1]{
  \posttitle{
    \begin{center}\large#1\end{center}
    }
}

\setlength{\droptitle}{-2em}
  \title{Using Prior Knowledge to Inform Future Restoration Activities in the
Gulf of Mexico}
  \pretitle{\vspace{\droptitle}\centering\huge}
  \posttitle{\par}
  \author{}
  \preauthor{}\postauthor{}
  \date{}
  \predate{}\postdate{}

\usepackage{lineno}
\linenumbers
\usepackage{setspace}
\linespread{2}
\usepackage{acronym}
\acrodef{tbep}[TBEP]{Tampa Bay Estuary Program}
\acrodef{chla}[chl-\textit{a}]{chlorophyll \textit{a}}

\begin{document}
\maketitle

\section{Introduction}\label{introduction}

Despite considerable investments over the last four decades in cCoastal
and estuarine ecosystem restoration (Diefenderfer et al.
\protect\hyperlink{ref-Diefenderfer16}{2016}) these efforts continue to
face many challenges that threaten to impede their success. In the Gulf
of Mexico, chronic and discrete drivers contribute to the difficulty in
restoring and managing coastal ecosystems. For example, the synergistic
effects of widespreadchronic coastal urbanization and climate change
impacts may limit habitat management effectiveness in the future
(Enwright et al. 2016). Competing management and policy directives for
flood protection, national commerce and energy development complicate
and prolong efforts to abate coastal hypoxia and other water quality
issues (Rabotyagov et al. 2014; OTHERS). Disputes surrounding fair and
equitable natural resource allocation often result in contentious
implementation plans for the long-term sustainability of coastal
resources (GMFMC 2017). And, discrete tropical storm (Greening et al.
2006; MORE RECENT?) and large-scale pollution events (Beyer et al. 2016)
often reset, reverse or delay progress in restoring coastal ecosystems.
These factors create a complex environment for successful implementation
of ecosystem restoration activities along the Gulf Coast.

Notwithstanding these challenges, the difficulty in rigorously
monitoring and understanding an ecosystem's condition and restoration
trajectory at various spatial and temporal scales further constrain a
recognition of restoration success (Hobbs and Harris 2001). The lack of
long-term environmental monitoring is a primary impediment to
understanding pre- versus post- restoration change (Schiff et al.
\protect\hyperlink{ref-Schiff16}{2016}) -- let alone identifying whether
prolonged management, policy and restoration activities have led to
actual improvements in coastal ecosystems. Where long-term coastal
monitoring programs have been implemented, a broader sense of how
management, policy and restoration activities affect coastal ecosystem
quality can be attained (Borja et al. 2015). New frameworks are starting
to emerge to better understand and facilitate the implementation of
society's practice of coastal restoration ecology from a more informed
perspective, utilizing lessons-learned from environmental monitoring
programs (Bayraktarov et al. 2016; Diefenderfer et al. 2016).

A very large, comprehensive and concerted effort to restore Gulf of
Mexico coastal ecosystems is currently underway (GCERC 2013, 2016).
Primary funding for this effort stems from the legal settlements
resulting from the 2010 Deepwater Horizon oil spill. Funding sources
include: early restoration investments that were made immediately
following the spill, resource damage assessments resulting from the
spill's impact (NRDA, 2016)), a record legal settlement of civil and
criminal penalties negotiated between the responsible parties and the US
government with strict US congressional oversight (RESTORE Act), and
matching funds from research, monitoring and restoration practitioners
worldwide. These funds, equating to \textgreater{}\$20B US, present the
Gulf of Mexico community an unprecedented opportunity to revitalize
regional restoration efforts that will span multiple generations (GCERC
2013, 2016). Consequently, the restoration investments being made with
these funds will be highly scrutinized to achieve successful outcomes.
Better understanding the environmental outcomes of past investments will
facilitate where and when future resources are implemented to have the
highest likelihood of achieving intended outcomes.

Because of the difficulties in demonstrating restoration success, new
tools are needed to help guide and support the implementation of Gulf of
Mexico restoration activities. Here, we present an empirical framework
for evaluating the success of investments in water quality improvement
activities to assign a probabilistic expectation of water quality
benefits. The framework synthesizes monitoring data across
spatiotemporal scales to demonstrate how the cumulative effects of
coastal restoration activities could improve water quality in an
estuary. Data on water quality and restoration projects in the Tampa Bay
area (Florida, USA) were used to demonstrate application of the analysis
framework. Tampa Bay is the second largest estuary in the Gulf of Mexico
and has been intensively monitored since the mid-1970s. The ecological
context of water quality changes in the Bay is well-described and a
comprehensive history of restoration projects occurring in the watershed
is available, making Tampa Bay an ideal test case for demonstrating the
evaluation framework. The water quality and restoration data were
evaluated to identify 1) types of restoration projects that produce the
greatest improvements in water quality, and 2) which time frames and
synergistic effects of projects are most relevant for having the largest
perceived benefits. Changes in chlorophyll concentrations as a proxy of
eutrophication were used to develop expectations of water quality
changes from restoration activities. The final product is a decision
support tool to evaluate alternative scenarios of implementation
strategies.

\section{Methods}\label{methods}

\subsection{Study area}\label{study-area}

Tampa Bay is located on the western coast of Florida and its watershed
is one of the most highly developed regions in Florida. More than 60
percent of land within 15 km of the Bay shoreline is a mix of urban and
suburban uses (SWFWD 2011). The Bay has been a focal point of economic
activity since the 1950s and currently supports a mix of industrial,
domestic, and recreational activities. The watershed includes one of the
largest phosphate production regions in the country, which is supported
by port operations primarily in the northeast portion of the Bay (H.
Greening and Janicki \protect\hyperlink{ref-Greening06}{2006}). Water
quality data have been collected since the 1970s when environmental
conditions were highly degraded. Nitrogen loads into the Bay in the
mid-1970s have been estimated as 8.2 x 106 kg year-1, most of which came
from untreated wastewater effluent. In addition to reduced aesthetics,
environmental conditions associated with hyper-eutrophy were common and
included elevated chlorophyll concentrations, increased occurrence of
harmful algal species, low concentrations of bottom water dissolved
oxygen, low water clarity, reduced seagrass coverage, and declines in
fishery yields for both sport and recreational species. Advanced
wastewater treatment operations were implemented at municipal plants by
the early 1980s. These efforts were successful in reducing nutrients
loads to the Bay by 90\%.

Current water quality in Tampa Bay is dramatically improved from
historical conditions. Most notably, seagrass coverage in 2016 was
reported as 16,785 hectares baywide, surpassing the restoration goal of
coverage in the 1950s (E. T. Sherwood et al.
\protect\hyperlink{ref-Sherwood17}{2017}). These changes have occured in
parallel with reductions in nutrient loading (Poe et al.
\protect\hyperlink{ref-Poe05}{2005}; H. Greening and Janicki
\protect\hyperlink{ref-Greening06}{2006}), chlorophyll concentrations
(Wang, Martin, and Morrison \protect\hyperlink{ref-Wang99}{1999}; Beck
and Hagy III \protect\hyperlink{ref-Beck15}{2015}), and improvements in
water clarity (Morrison et al. \protect\hyperlink{ref-Morrison06}{2006};
Beck, Hagy III, and Le \protect\hyperlink{ref-Beck17c}{2017}). Most of
these positive changes have resulted from management efforts to reduct
point source controls on nutrient pollution in the highly developed
areas of Hillsborough Bay (Johansson
\protect\hyperlink{ref-Johansson91}{1991}; Johansson and Lewis III
\protect\hyperlink{ref-Johansson92}{1992}). However, the cumulative
impacts of over 500 additional management activities have likely
contributed to improvements in water quality over time. Projects from
both public and private entities have been completed since the 1980s.
These projects represent numerous voluntary (e.g., habitat acquisition,
restoration, watershed retention, etc.) and compliance-driven (e.g.,
site-level permitting, power plant scrubber upgrades, etc.) activities.
Although it is generally recognized that these projects have contributed
to overall Bay improvements, the cumulativ effects relative to
large-scale management efforts are not well understood. Understanding
the impacts within relevant spatial boundaries and how these projects
have jointly contributed to changes over time will provide an improved
understanding of the link between Bay improvements and restoration
activities.

\subsection{Data sources}\label{data-sources}

In addition to legacy improvements at wastewater treatment plants, over
500 restoration projects have been documented in the Tampa Bay area
since 1971. Two databases were synthesized to provide a comprehensive
history of projects that have occurred in Tampa Bay and its watershed.
The first dataset was obtained from the Tampa Bay Water Atlas (version
2.3, \url{http://maps.wateratlas.usf.edu/tampabay/}, TBEP (Tampa Bay
Estuary Program) (\protect\hyperlink{ref-TBEP17}{2017})) maintained as a
joint resource by the University of South Florida and the \ac{tbep}.
This database included 253 projects from 1971 to 2007 that were
primarily focused on habitat establishment, restoration, or protection
in the nearshore areas of the Bay or the larger watershed (e.g.,
planting of \emph{Spartina alterniflora}, exotic vegetation control,
etc.). The database was limited to basic information, such as year of
completion, geographic coordinates, general activities, and areal
coverage. The second database was obtained from the \ac{tbep} Action
Plan Database Portal (\url{https://apdb.tbeptech.org/index.php}) to
describe locations of infrastructure improvement projects. This database
included 334 projects from 1992 to 2016 for county or municipal
activities, such as implementation of best management practices at
treatment plants, creation of stormwater retention or treatment
controls, or site-specific controls of point sources.

Both data sources were combined to provide a single dataset describing
the location, year of completion, and project classification. The
project classifications were described in two nested categories. The
first described a high-level classification for each project as habitat
or water infrastructure. The second was a lower-level classification for
habitat projects as enhancement, establishment, and protection and water
infrastructure projects as non-point source or point source controls.
These categories were used to provide a broad characterization of
restoration activites that were considered relevant for the perceived
improvements in water quality over time. The nested categories were used
to develop separate probabilistic models describing the likelihood of
changes in water quality (described below). The final dataset included
571 projects from 1971 to 2016. Projects with incomplete information
(i.e., missing date) were not included in the final dataset.

Water quality data in Tampa Bay have been consistently collected since
1974 by the Hillsborough County Environmental Protection Commission
(TBEP (Tampa Bay Estuary Program) \protect\hyperlink{ref-TBEP17}{2017}).
Data were collected monthly at forty-five stations using a water sample
from mid-depth or a monitoring sonde depending on the parameter.
Monitoring stations are fixed in a grid covering the entire bay from the
northern oligohaline sections to the opening with the Gulf of Mexico.
Water samples at each station are processed immediately after
collection. Measurements at each site include temperature (C), secchi
depth (m), dissolved oyxgen (mg/L), conductivity (\(\mu\)Ohms/cm), pH,
salinity (psu), turbidity (Nephalometric Turbidy Units), \ac{chla}
(\(\mu\)g/L), total nitrogen (mg/L), and total phosphorus (mg/L). For
the probabilistic models, all measurements of salinity, total nitrogen,
and \ac{chla} were combind for a total of 515 monthly observations of
each parameter at each station.

\subsection{Data synthesis and analysis
framework}\label{data-synthesis-and-analysis-framework}

Combining the restoration and water quality datasets was a critical part
of developing the analysis framework. Each dataset described events or
sampling activities with unique dates and locations and simple pairing
of restoration projects with water quality data was impractical. To
address this challenge, observations in each dataset were spatially and
temporally matched using an approach to maximize the potential of
identifying a unique effect of the restoration projects on changes in
water quality.

The matching between the two datasets began with a spatial join where
the Euclidean distances between each water quality station and each
restoration project were quantified. The spatial matches were used to
create a ranking of project sites from each water quality station based
on distance. The distances were also grouped by the five restoration
project types (i.e., habitat protection, non-point source control, etc.)
such that the closest \(n\) sites of a given project type could be
identified for any water quality station. Temporal matching between
water quality stations and restoration projects was obtained by
subsetting the water quality data within a time window before and after
the completion date of each spatially-matched restoration project. For
the closest \(n\) restoration sites for each of five project types, two
summarized water quality estimates were obtained to quantify a before
and after effect of each project. Time windows that overlapped the start
and end date of the water quality time series were discarded. The final
two estimates of the before and after effects of the five types of
restoration projectsat each water quality station were based on an
average of the \(n\) closest restoration sites, weighted inversely by
distance from the monitoring station.

The combined water quality and restoration data were used as input for
developing the probabilistic models. Two parameters affected the
synthesis of the datasets which directly controlled the ability to
identify an effect of each restoration project type, 1) the number of
spatially-matched restoration projects used to average the cumulative
effect of each project type, and 2) the time windows before and after a
project completion date that were used to subset each water quality time
series. Identifying the two values that maximized the difference between
before and after water quality measurements was necessary to quantify
how many projects induced a change in water quality, the time within
which a change is expected, and the magnitude of a change differed
between project types.

Bayesian network, simple models pre/post, large model whole record.

\section{Results}\label{results}

\section{Discussion}\label{discussion}

\section*{References}\label{references}
\addcontentsline{toc}{section}{References}

\hypertarget{refs}{}
\hypertarget{ref-Beck15}{}
Beck, M. W., and J. D. Hagy III. 2015. ``Adaptation of a Weighted
Regression Approach to Evaluate Water Quality Trends in an Estuary.''
\emph{Environmental Modelling and Assessment} 20 (6): 637--55.
doi:\href{https://doi.org/10.1007/s10666-015-9452-8}{10.1007/s10666-015-9452-8}.

\hypertarget{ref-Beck17c}{}
Beck, M. W., J. D. Hagy III, and C. Le. 2017. ``Quantifying Seagrass
Light Requirements Using an Algorithm to Spatially Resolve Depth of
Colonization.'' \emph{Estuaries and Coasts}, 1--17.

\hypertarget{ref-Diefenderfer16}{}
Diefenderfer, Heida L., Gary E. Johnson, Ronald M. Thom, Kate E. Buenau,
Laurie A. Weitkamp, Christa M. Woodley, Amy B. Borde, and Roy K. Kropp.
2016. ``Evidence-Based Evaluation of the Cumulative Effects of Ecosystem
Restoration.'' \emph{Ecosphere} 7 (3): e01242.
\url{http://dx.doi.org/10.1002/ecs2.1242}.

\hypertarget{ref-Greening06}{}
Greening, H., and A. Janicki. 2006. ``Toward Reversal of Eutrophic
Conditions in a Subtrophical Estuary: Water Quality and Seagrass
Response to Nitrogen Loading Reductions in Tampa Bay, Florida, USA.''
\emph{Environmental Management} 38 (2): 163--78.

\hypertarget{ref-Johansson91}{}
Johansson, J. O. R. 1991. ``Long-Term Trends of Nitrogen Loading, Water
Quality and Biological Indicators in Hillsborough Bay, Florida.'' Edited
by S. F. Treat and P. A. Clark. Tampa, Florida, USA: Tampa Bay Area
Study Group Project at Scholar Commons, 157--76.

\hypertarget{ref-Johansson92}{}
Johansson, J. O. R., and R. R. Lewis III. 1992. ``Recent Improvements in
Water Quality and Biological Indicators in Hillsborough Bay, a Highly
Impacted Subdivision of Tampa Bay, Florida, USA.'' \emph{Marine Coastal
Eutrophication} Proceedings of an International Conference, Bologna,
Italy, 21-24 March 1990: 1199--1215.

\hypertarget{ref-Morrison06}{}
Morrison, G., E. T. Sherwood, R. Boler, and J. Barron. 2006.
``Variations in Water Clarity and Chlorophyll\emph{a} in Tampa Bay,
Florida, in Response to Annual Rainfall, 1985-2004.'' \emph{Estuaries
and Coasts} 29 (6): 926--31.

\hypertarget{ref-Poe05}{}
Poe, A., K. Hackett, S. Janicki, R. Pribble, and A. Janicki. 2005.
``Estimates of Total Nitrogen, Total Phosphorus, Total Suspended Solids,
and Biochemical Oxygen Demand Loadings to Tampa Bay, Florida:
1999-2003.'' \#02-05. St. Petersburg, Florida, USA: Tampa Bay Estuary
Program.

\hypertarget{ref-Schiff16}{}
Schiff, K., P.R. Trowbridge, E.T. Sherwood, P. Tango, and R.A. Batiuk.
2016. ``Regional Monitoring Programs in the United States: Synthesis of
Four Case Studies from Pacific, Atlantic, and Gulf Coasts.''
\emph{Regional Studies in Marine Science} 4: A1--A7.
\url{https://doi.org/10.1016/j.rsma.2015.11.007}.

\hypertarget{ref-Sherwood17}{}
Sherwood, E. T., H. S. Greening, J. O. R. Johansson, K. Kaufman, and G.
Raulerson. 2017. ``Tampa Bay (Florida, USA): Documenting Seagrass
Recovery Since the 1980s and Reviewing the Benefits.''
\emph{Southeastern Geographer} 57 (3): 294--319.

\hypertarget{ref-TBEP17}{}
TBEP (Tampa Bay Estuary Program). 2017. ``Tampa Bay Water Atlas.''

\hypertarget{ref-Wang99}{}
Wang, P. F., J. Martin, and G. Morrison. 1999. ``Water Quality and
Eutrophication in Tampa Bay, Florida.'' \emph{Estuarine, Coastal and
Shelf Science} 49 (1): 1--20.


\end{document}
