\documentclass[]{article}
\usepackage{lmodern}
\usepackage{amssymb,amsmath}
\usepackage{ifxetex,ifluatex}
\usepackage{fixltx2e} % provides \textsubscript
\ifnum 0\ifxetex 1\fi\ifluatex 1\fi=0 % if pdftex
  \usepackage[T1]{fontenc}
  \usepackage[utf8]{inputenc}
\else % if luatex or xelatex
  \ifxetex
    \usepackage{mathspec}
  \else
    \usepackage{fontspec}
  \fi
  \defaultfontfeatures{Ligatures=TeX,Scale=MatchLowercase}
\fi
% use upquote if available, for straight quotes in verbatim environments
\IfFileExists{upquote.sty}{\usepackage{upquote}}{}
% use microtype if available
\IfFileExists{microtype.sty}{%
\usepackage{microtype}
\UseMicrotypeSet[protrusion]{basicmath} % disable protrusion for tt fonts
}{}
\usepackage[margin=1in]{geometry}
\usepackage{hyperref}
\PassOptionsToPackage{usenames,dvipsnames}{color} % color is loaded by hyperref
\hypersetup{unicode=true,
            pdftitle={Assessment of the cumulative effects of restoration activities on water quality in Tampa Bay, Florida},
            colorlinks=true,
            linkcolor=Maroon,
            citecolor=Blue,
            urlcolor=blue,
            breaklinks=true}
\urlstyle{same}  % don't use monospace font for urls
\usepackage{graphicx,grffile}
\makeatletter
\def\maxwidth{\ifdim\Gin@nat@width>\linewidth\linewidth\else\Gin@nat@width\fi}
\def\maxheight{\ifdim\Gin@nat@height>\textheight\textheight\else\Gin@nat@height\fi}
\makeatother
% Scale images if necessary, so that they will not overflow the page
% margins by default, and it is still possible to overwrite the defaults
% using explicit options in \includegraphics[width, height, ...]{}
\setkeys{Gin}{width=\maxwidth,height=\maxheight,keepaspectratio}
\IfFileExists{parskip.sty}{%
\usepackage{parskip}
}{% else
\setlength{\parindent}{0pt}
\setlength{\parskip}{6pt plus 2pt minus 1pt}
}
\setlength{\emergencystretch}{3em}  % prevent overfull lines
\providecommand{\tightlist}{%
  \setlength{\itemsep}{0pt}\setlength{\parskip}{0pt}}
\setcounter{secnumdepth}{0}
% Redefines (sub)paragraphs to behave more like sections
\ifx\paragraph\undefined\else
\let\oldparagraph\paragraph
\renewcommand{\paragraph}[1]{\oldparagraph{#1}\mbox{}}
\fi
\ifx\subparagraph\undefined\else
\let\oldsubparagraph\subparagraph
\renewcommand{\subparagraph}[1]{\oldsubparagraph{#1}\mbox{}}
\fi

%%% Use protect on footnotes to avoid problems with footnotes in titles
\let\rmarkdownfootnote\footnote%
\def\footnote{\protect\rmarkdownfootnote}

%%% Change title format to be more compact
\usepackage{titling}

% Create subtitle command for use in maketitle
\newcommand{\subtitle}[1]{
  \posttitle{
    \begin{center}\large#1\end{center}
    }
}

\setlength{\droptitle}{-2em}

  \title{Assessment of the cumulative effects of restoration activities on water
quality in Tampa Bay, Florida}
    \pretitle{\vspace{\droptitle}\centering\huge}
  \posttitle{\par}
    \author{}
    \preauthor{}\postauthor{}
    \date{}
    \predate{}\postdate{}
  
\usepackage{lineno}
\linenumbers
\usepackage{setspace}
\linespread{2}
\usepackage{cleveref}
\usepackage{acronym}
\usepackage{makecell}
\acrodef{tbep}[TBEP]{Tampa Bay Estuary Program}
\acrodef{chla}[chl-\textit{a}]{chlorophyll \textit{a}}

\begin{document}
\maketitle

\hypertarget{introduction}{%
\section{Introduction}\label{introduction}}

Despite considerable investments over the last four decades in coastal
and estuarine ecosystem restoration (Diefenderfer et al.
\protect\hyperlink{ref-Diefenderfer16}{2016}), numerous challenges still
impede comprehensive success. In the Gulf of Mexico (GOM), chronic and
discrete drivers contribute to the difficulty in restoring and managing
coastal ecosystems. For example, the synergistic effects of widespread
chronic coastal urbanization and climate change impacts will likely
limit future habitat management effectiveness (Enwright, Griffith, and
Osland \protect\hyperlink{ref-Enwright16}{2016}). Competing management
and policy directives for flood protection, national commerce and energy
development complicate and prolong efforts to abate coastal hypoxia and
other coastal water quality issues (Rabotyagov et al.
\protect\hyperlink{ref-Rabotyagov14}{2014}; Alfredo and Russo
\protect\hyperlink{ref-Alfredo17}{2017}). Disputes surrounding fair and
equitable natural resource allocation often result in contentious
implementation plans for the long-term sustainability of coastal
resources (GMFMC 2017). And, discrete tropical storm (Greening, Doering,
and Corbett \protect\hyperlink{ref-Greening06b}{2006}) and large-scale
pollution events (Beyer et al. \protect\hyperlink{ref-Beyer16}{2016})
often reset, reverse or delay progress in restoring coastal ecosystems.
These factors contribute to a complex setting for successful
implementation of ecosystem restoration activities within the GOM.

Notwithstanding these challenges, the difficulty in rigorously
monitoring and understanding an ecosystem's condition and restoration
trajectory at various spatial and temporal scales further constrain
rigorous observations of restoration success (Hobbs and Harris
\protect\hyperlink{ref-Hobbs01}{2001}). The lack of long-term
environmental monitoring is a primary impediment to understanding pre-
versus post- restoration change (Schiff et al.
\protect\hyperlink{ref-Schiff16}{2016}) -- while also impeding the
recognition of any coastal ecosystem improvements derived from prolonged
management, policy and restoration activities. Where long-term coastal
monitoring programs have been implemented, a broader sense of how
management, policy and restoration activities affect coastal ecosystem
quality can be attained (Borja et al.
\protect\hyperlink{ref-Borja16}{2016}). Utilizing lessons-learned from
environmental monitoring programs, new frameworks are starting to emerge
to better understand and facilitate the implementation of coastal
restoration ecology from a more informed perspective (Bayraktarov et al.
\protect\hyperlink{ref-Bayraktarov16}{2016}; Diefenderfer et al.
\protect\hyperlink{ref-Diefenderfer16}{2016}).

A very large, comprehensive and concerted effort to restore Gulf of
Mexico coastal ecosystems is currently underway (GCERC 2013, 2016).
Primary funding for this effort stems from the legal settlements
resulting from the 2010 Deepwater Horizon oil spill. Funding sources
include: early restoration investments that were made immediately
following the spill, resource damage assessments resulting from the
spill's impacts (NRDA, 2016), a record legal settlement of civil and
criminal penalties negotiated between the responsible parties and the US
government with strict US congressional oversight (RESTORE Act), and
matching funds from research, monitoring and restoration practitioners
worldwide. These funds, equating to \textgreater{}\$20B US, present the
Gulf of Mexico community an unprecedented opportunity to revitalize
regional restoration efforts that will span multiple generations (GCERC
2013, 2016). Consequently, the restoration investments being made with
these funds will be highly scrutinized. Better understanding the
environmental outcomes of past restoration investments will help
identify how, where and when future resources should be invested so that
the Gulf Coast community can achieve the highest degree of restoration
success.

Because of the difficulties in demonstrating restoration success, new
tools are needed to help guide and support the implementation of GOM
restoration activities. Here, we present an empirical framework for
evaluating the influence of multiple restoration project types on water
quality improvements within a GOM estuary. The framework helps
synthesize routine, ambient monitoring data across various
spatio-temporal scales to demonstrate how the cumulative effects of
coastal restoration activities contribute towards broad estuarine water
quality improvements. Data on water quality and restoration projects in
the Tampa Bay area (Florida, USA) were used to demonstrate application
of the analysis framework. Tampa Bay is the second largest estuarine
embayment in the GOM and has been intensively monitored since the
mid-1970s. The ecological context of water quality changes in the Bay is
well-described (Greening et al.
\protect\hyperlink{ref-Greening2014}{2014}) and a comprehensive history
of restoration projects occurring in the watershed is available, making
Tampa Bay an ideal test case for demonstrating and applying a new
evaluation framework. The water quality and restoration datasets were
evaluated to identify: 1) the types of restoration activities that
contribute to the greatest improvements in water quality, and 2) the
time frames over which water quality benefits may be elucidated from
synergistic restoration activities. Changes in chlorophyll-a
concentrations, a proxy for negative eutrophication effects within Tampa
Bay (Greening et al. \protect\hyperlink{ref-Greening2014}{2014}), were
used as the success metric to evaluate estuarine restoration activities.
The final product is an open-source, decision support tool that will
help restoration practitioners evaluate alternative scenarios for
implementing future restoration strategies.

\hypertarget{methods}{%
\section{Methods}\label{methods}}

\hypertarget{study-area}{%
\subsection{Study area}\label{study-area}}

Tampa Bay is located on the west-central GOM coast of the Florida
peninsula, and its watershed is one of the most highly developed regions
in Florida (\cref{fig:map}). More than 60 percent of land within 15 km
of the Bay shoreline is a mix of urban and suburban uses (SWFWMD 2011).
The Bay has been a focal point of economic activity since the 1950s and
currently supports a mix of industrial, domestic, and recreational
activities. The watershed includes one of the largest phosphate
production regions in the country, which is supported by port operations
primarily in the northeast portion of the Bay (Greening et al.
\protect\hyperlink{ref-Greening2014}{2014}). Water quality data have
been collected since the 1970s when environmental conditions were highly
degraded. Nitrogen loads into the Bay in the mid-1970s have been
estimated as 8.9 x 10\^{}6 kg year-1, most of which came from untreated
wastewater effluent (Greening et al.
\protect\hyperlink{ref-Greening2014}{2014}). In addition to reduced
aesthetics, hypereutrophic environmental conditions were common and
included elevated chlorophyll-a concentrations, increased occurrence of
harmful algal species, low concentrations of bottom water dissolved
oxygen, low water clarity, reduced seagrass coverage, and reported
declines in fishery yields for both sport and recreational species.
Advanced wastewater treatment operations were implemented at municipal
plants by the early 1980s. These efforts were successful in reducing
nutrients loads to the Bay by as much as 90\%.

\begin{figure}
\centerline{\includegraphics[width = 0.8\textwidth]{figs/tbrest_map.pdf}}
\caption{Water quality stations and restoration projects in the Tampa Bay area.  Water quality stations have been monitored monthly since 1974.  Locations of restoration projects represent 887 records that are generally categorized as habitat or water infrastructure projects from 1971 to present.  Bay segments as management units of interest are shown in the upper right inset. HB: Hillsborough Bay, LTB: Lower Tampa Bay, MTB: Middle Tampa Bay, OTB: Old Tampa Bay.}
\label{fig:map}
\end{figure}

Current water quality in Tampa Bay is dramatically improved from
historical conditions. Most notably, seagrass coverage in 2016 was
reported as 16,857 hectares baywide, surpassing the restoration goal of
coverage in the 1950s (Sherwood et al.
\protect\hyperlink{ref-Sherwood17}{2017}). These changes have occurred
in parallel with reductions in nutrient loading (Poe et al.
\protect\hyperlink{ref-Poe05}{2005}; Greening et al.
\protect\hyperlink{ref-Greening2014}{2014}), chlorophyll concentrations
(Wang, Martin, and Morrison \protect\hyperlink{ref-Wang99}{1999}; Beck
and Hagy III \protect\hyperlink{ref-Beck15}{2015}), and improvements in
water clarity (Morrison et al. \protect\hyperlink{ref-Morrison06}{2006};
Beck, Hagy III, and Le \protect\hyperlink{ref-Beck17c}{2017}). Most of
these positive changes have resulted from management efforts to reduce
point source controls on nutrient pollution in the highly developed
areas of Hillsborough Bay (Johansson
\protect\hyperlink{ref-Johansson91}{1991}; Johansson and Lewis III
\protect\hyperlink{ref-Johansson92}{1992}).

The cumulative and synergistic effects of nearly 900 additional
management activities have also likely contributed to water quality
improvements over the past 4 decades, yet no previous efforts have been
made to quantify potential associations between these projects and water
quality. Several hundred projects from both public and private entities
have been completed since the 1971. These projects represent numerous
voluntary (e.g., coastal habitat acquisition, restoration, preservation,
etc.) and compliance-driven (e.g., stormwater retrofits, process water
treatment upgrades, site-level permitting, power plant scrubber
upgrades, improved agricultural practices, residential fertilizer use
ordinances, etc.) activities. Although it is generally recognized that
these projects have contributed to overall estuarine ecosystem
improvements, their cumulative effects, relative to broad
watershed-scale management efforts, are not well understood.
Understanding how these projects affect adjacent estuarine water quality
at various spatio-temporal scales will provide an improved understanding
of the link between overall estuary improvements and specific
restoration activities.

\hypertarget{data-sources}{%
\subsection{Data sources}\label{data-sources}}

Several databases were synthesized to provide a comprehensive history of
restoration projects that have occurred in Tampa Bay and its watershed.
The first dataset was obtained from the Tampa Bay Water Atlas (version
2.3, \url{http://maps.wateratlas.usf.edu/tampabay/}, TBEP (Tampa Bay
Estuary Program) (\protect\hyperlink{ref-TBEP17}{2017})) maintained as a
joint resource by the University of South Florida, the \ac{tbep}, and
partners. This database included 253 projects from 1971 to 2007 that
were primarily focused on habitat establishment, enhancement, or
protection along the Bay's immediate shoreline or within the larger
watershed area (e.g., restoration of salt marshes and mangroves, exotic
vegetation control, conversion of agricultural lands to natural
habitats, etc.). Information on more recent projects (2008-2017)
acquired from the US EPA's National Estuary Program Mapper
(\url{https://gispub2.epa.gov/NEPmap/}) included an additional 265
projects. This database was limited to basic information, such as year
of completion, geographic coordinates, general activities, and areal
coverage. The last database was obtained from the \ac{tbep} Action Plan
Database Portal (\url{https://apdb.tbeptech.org/index.php}) to describe
locations of broader infrastructure improvement projects, structural
best management practices, and policy-driven management actions. This
database included 368 projects from 1992 to 2016 for county, municipal
or industrial activities, such as implementation of best management
practices at treatment plants, creation of stormwater retention or
treatment controls, or site-specific controls of point sources.

For all restoration datasets, shared information included the location,
year of completion, and project classification of the restoration
activity. Because the types of projects differed, a classification
scheme was developed that first described projects broadly as habitat or
water infrastructure improvements and secondarily as a lower-level
classification for habitat projects: enhancement, establishment, and
protection; and water infrastructure projects: nonpoint source or point
source controls. These categories were used to provide a broad
characterization of restoration activities that were considered relevant
for the perceived improvements in water quality over time. The five
sub-categories (habitat enhancement, establishment, and protection;
non-point and point source controls) were separately evaluated to
describe the likelihood of changes in water quality associated with each
type (described below). The final combined dataset included 887 projects
from 1971 to 2017 (\cref{fig:restyrs}). Projects with incomplete
information (i.e., missing date) were not included in the final dataset.

\begin{figure}
\centerline{\includegraphics[width = \textwidth]{figs/restyrs.pdf}}
\caption{Counts (top) and locations (bottom) of restoration project types over time in the Tampa Bay watershed.  Restorations were categorized as water infrastructure (blue; nonpoint source controls, point source controls) and habitat (green; enhancements, establishments, protection) projects.  The compiled restoration database included records of 887 project types and locations from 1971 to 2017.}
\label{fig:restyrs}
\end{figure}

Water quality data in Tampa Bay have been consistently collected since
1974 by the Environmental Protection Commission of Hillsborough County
(Sherwood et al. \protect\hyperlink{ref-Sherwood16}{2016}; TBEP (Tampa
Bay Estuary Program) \protect\hyperlink{ref-TBEP17}{2017}). Data were
collected monthly at forty-five stations using a water sample from
mid-depth or a monitoring sonde depending on the parameter. Monitoring
stations are fixed and cover the entire bay from the uppermost
mesohaline sections to the lowermost euhaline portions that have direct
interaction with the GOM. Water samples at each station are laboratory
processed immediately after collection. Measurements at each site
include temperature (\textsuperscript{o}C), Secchi disk depth (m),
dissolved oxygen (mg/L), conductivity (\(\mu\)Ohms/cm), pH, salinity
(psu), turbidity (Nephalometric Turbidity Units), \ac{chla}
(\(\mu\)g/L), total nitrogen (mg/L), and total phosphorus (mg/L). For
the models, all measurements of salinity, total nitrogen, and \ac{chla}
were combined for a total of 515 monthly observations of each parameter
at each station.

\hypertarget{data-synthesis-and-analysis-framework}{%
\subsection{Data synthesis and analysis
framework}\label{data-synthesis-and-analysis-framework}}

Combining the restoration and water quality datasets was a critical part
of developing the analysis framework. Each dataset described events or
sampling activities with unique dates and locations and a simple
synoptic pairing of restoration projects with water quality data was not
possible. To address this challenge, observations in each dataset were
spatially and temporally matched using an approach designed to maximize
the potential of identifying a unique effect of the restoration projects
on changes in water quality. Water quality monitoring sites were matched
to the closest selected restoration projects and changes in the water
quality data were evaluated relative to the completion dates of the
selected projects.

\begin{figure}
\includegraphics[width=\textwidth]{figs/spmtch} \caption{Spatial matching of water quality stations with restoration projects. Spatial matches of each water quality station (blue dots) with habitat (solid line to grey dots) and water infrastucture (dashed line to black dots) projects are shown as the closest single match by type on the left and the "n" closest matches on the right.  The spatial matches were made for the five restoration project types within the broader habitat and water categories shown in the figure.}\label{fig:spmtch}
\end{figure}

\begin{figure}
\includegraphics[width=\textwidth]{figs/tmmtch} \caption{Temporal matching of restoration project types with time series data at a water quality station.  The restoration project locations that are spatially matched with a water quality station (a) are used to create a temporal slice of the water quality data within a window of time before and after the completion date of each restoration project (b).  Slices are based on the closest "n" restoration projects by type (n = 2 in this example) to a water quality station.  The two broad categories of habitat and water infrastructure projects are shown in the figure as an example, whereas the analysis evaluated all five restoration sub-categories.}\label{fig:tmmtch}
\end{figure}

The matching between the two datasets began with a spatial join where
the Euclidean distances between each water quality station and each
restoration project were quantified. The spatial matches were used to
create a ranking of project sites from each water quality station based
on distance. The distances were also grouped by the five restoration
project types (i.e., habitat protection, nonpoint source control, etc.)
such that the closest \(n\) sites of a given project type could be
identified for any water quality station (\cref{fig:spmtch}).

For each spatial match, temporal matching between water quality stations
and restoration projects was obtained by subsetting the water quality
data within a time window before and after the completion date of each
restoration project (\cref{fig:tmmtch}). For the closest \(n\)
restoration sites for each of five project types, two summarized water
quality estimates were obtained to quantify a before and after estimate
of chlorophyll associated with each project. Time windows that
overlapped the start and end date of the water quality time series were
discarded. The final two estimates of the before and after effects of
the five types of restoration projects at each water quality station
were based on an average of the \(n\) closest restoration sites,
weighted inversely by distance from the monitoring station.

Change in water quality relative to each type of restoration project was
estimated as: \begin{equation}
\delta WQ = \frac{\sum_{i = 1}^{n} \hat{wq} \in win + proj_{i, dt}}{n \cdot dist_{i \in n}} - \frac{\sum_{i = 1}^{n} \hat{wq} \in proj_{i, dt} - win}{n \cdot dist_{i \in n}}
\label{eq:wqdif}
\end{equation} where \(\delta WQ\) was the difference between the after
and before averages for each of \(n\) spatially matched restoration
projects. For each \(i\) of \(n\) projects (\(proj\)), the average water
quality (\(\hat{wq}\)) within the window (\(win\)) either before
(\(proj_{i, dt} - win\)) or after (\(win + proj_{i, dt}\)) the
completion date (\(dt\)) for project \(i\) was summed. The summation of
water quality before and after each project was then divided by the
total number of \(n\) matched projects, multiplied by the distance of
the projects from a water quality station (\(dist_{i \in n}\)). This
created a weighted average of the before-after effects of each project
that was inversely related to the distance from a water quality station.
A weighted average by distance was used based on the assumption that
restoration projects farther from a water quality station will have a
weaker association with potential changes in chlorophyll. The total
change in water quality for a project type was simply the difference in
weighted averages. This process was repeated for every station and a
graphical example of \cref{eq:wqdif} is shown in \cref{fig:statex}

\begin{figure}
\centerline{\includegraphics[width = 0.95\textwidth]{figs/statex.pdf}}
\caption{Steps to estimate cumulative effects of water quality changes at a single station relative to a selected number of projects and time windows. Subplot (a) shows station 23 in Middle Tampa Bay matched to the five nearest restoration projects for each of five types.  The time slices of the water quality observations for +/- ten years before and after the completion of each project are shown in (b), ordered from near to far.  The before/after water quality averages for the slices are shown in (c) and the differences between the two are shown in (d).  Finally, the weighted averages for the five closest matches by project type are shown in (e) with 95\% confidence intervals. }
\label{fig:statex}
\end{figure}

There are key assumptions made by the above approach regarding how the
analysis was conducted and what information is obtained from the result.
First, our spatial-temporal matching of water quality data with
restoration projects is strictly associative where the general
assumption is that restoration projects will benefit water quality
through a decrease in chlorophyll. We make no assumptions about the
expected magnitude of an association given that the model does not
describe a mechanism of change. However, a general expectation is that
chlorophyll changes will be different by project type and this
association will vary as a function of distance and evaluated time
windows. An expected outcome is that qualitative statements can be made
about the relative differences between projects types, particularly
regarding how more projects of a particular type could benefit water
quality and within what general time windows a change might be expected.

An additional assumption is that the model was designed to describe
cumulative effects at different spatial scales. In \cref{eq:wqdif}, the
association of a restoration type with chlorophyll is estimated for one
water quality station, whereas estimates from several water quality
stations can be combined to develop an overall description of a
particular restoration type as it applies to an areal unit of interest.
For example, estimated associations of point source control projects
with each water quality station in the bay can be combined to develop an
overall narrative of how these projects could positively effect
environmental change in the bay. The examples in
\cref{fig:spmtch,fig:tmmtch,fig:statex} are focused on individual
stations to demonstrate core principles of the approach. Estimates
across stations were evaluated to describe baywide effects of
restoration project types and by individual bay segments that have
specific management targets for chlorophyll concentration (Florida
Statute 62-302.532, Janicki, Wade, and Pribble
\protect\hyperlink{ref-Janicki99}{1999}). This approach was used given
the uneven distribution of restoration projects in space and time
relative to the known changes in water quality that has been documented
in the different Bay segments (Greening et al.
\protect\hyperlink{ref-Greening2014}{2014}).

Finally, parameters in \cref{eq:wqdif} affected the synthesis of the
datasets which directly controlled the ability to characterize
associations of each restoration project type with water quality
changes, 1) \(n\), the number of spatially-matched restoration projects
used to average the cumulative effect of each project type, and 2)
\(win\), the time windows before and after a project completion date
that were used to subset each water quality time series. Identifying
values that maximized the difference between before and after water
quality measurements was necessary to quantify how many projects were
most strongly associated with a change in water quality, the time within
which a change is expected, and the magnitude of an expected change
between project types. All analyses were conducted with the R
statistical programming language (RDCT (R Development Core Team)
\protect\hyperlink{ref-RDCT18}{2018}).

\hypertarget{results}{%
\section{Results}\label{results}}

\hypertarget{observed-data}{%
\subsection{Observed data}\label{observed-data}}

Observed water quality trends in Tampa Bay showed a long-term decrease
in Chlorophyll-a over the forty-year record consistent with documented
changes (Wang, Martin, and Morrison
\protect\hyperlink{ref-Wang99}{1999}; Greening et al.
\protect\hyperlink{ref-Greening2014}{2014}; Beck and Hagy III
\protect\hyperlink{ref-Beck15}{2015}) (\cref{tab:statsum}). Median
concentrations were highest in the earlier period of record from 1977 to
1987 (median 13.40 \(\mu\)g/L at low salinity stations, 7.30 \(\mu\)g/L
at high salinity stations). Declines were consistent throughout the
period of record with the largest reductions occurring during the first
twenty years (34\% decrease), followed by consistent but smaller
reductions in concentrations later in the time series. A 34\% decrease
at low salinity stations and a 30\% decrease at high salinity stations
was observed between the periods of 1977-1987 to 1987-1997. Seasonally,
chlorophyll concentrations were highest in the late summer/early fall
periods (median 13.80 \(\mu\)g/L at low salinity stations, 7.23
\(\mu\)g/L at high salinity stations, across all years). Similarly,
total nitrogen concentrations had similar trends as chlorophyll,
although a consistent decline similar in magnitude was observed across
all four decades (Poe et al. \protect\hyperlink{ref-Poe05}{2005};
Greening et al. \protect\hyperlink{ref-Greening2014}{2014}). An
exception for nitrogen was observed at high salinity stations where
concentrations were relatively constant at approximately 0.55 mg/L from
1987 to 2007. Seasonally, nitrogen peaked in the late summer/early fall
period.

\begin{table}[!tbp]
\caption{Summary of total nitrogen and chlorophyll-a observations from monitoring stations in Tampa Bay.  Minimum, median, and maximum observed values for low and high salinity conditions are shown for seasonal and annual aggregations of water quality observations at all monitoring stations (See \cref{fig:map}).  Low or high salinity is based on values below or above the long-term baywide median (26.5 psu). JFM: January, February, March; AMJ: April, May, June; JAS: July, August, September; OND: October, November, December.\label{tab:statsum}} 
\begin{center}
\begin{tabular}{llllclll}
\hline\hline
\multicolumn{1}{l}{\bfseries Time period}&\multicolumn{3}{c}{\bfseries Total nitrogen}&\multicolumn{1}{c}{\bfseries }&\multicolumn{3}{c}{\bfseries Chlorophyll-a}\tabularnewline
\cline{2-4} \cline{6-8}
\multicolumn{1}{l}{}&\multicolumn{1}{c}{Min}&\multicolumn{1}{c}{Median}&\multicolumn{1}{c}{Max}&\multicolumn{1}{c}{}&\multicolumn{1}{c}{Min}&\multicolumn{1}{c}{Median}&\multicolumn{1}{c}{Max}\tabularnewline
\hline
{\bfseries Low}&&&&&&&\tabularnewline
~~JFM&$0.00$&$0.46$&$2.69$&&$0.12$&$ 5.30$&$114.40$\tabularnewline
~~AMJ&$0.03$&$0.59$&$3.03$&&$0.20$&$ 8.40$&$183.40$\tabularnewline
~~JAS&$0.02$&$0.64$&$3.02$&&$0.50$&$13.80$&$266.60$\tabularnewline
~~OND&$0.03$&$0.57$&$4.14$&&$0.00$&$10.00$&$192.14$\tabularnewline
~~1977-1987&$0.02$&$0.88$&$3.03$&&$0.10$&$13.40$&$266.60$\tabularnewline
~~1987-1997&$0.05$&$0.73$&$4.14$&&$0.00$&$ 8.78$&$192.14$\tabularnewline
~~1997-2007&$0.00$&$0.54$&$2.89$&&$0.12$&$ 7.86$&$261.90$\tabularnewline
~~2007-2017&$0.03$&$0.42$&$2.75$&&$0.50$&$ 7.40$&$220.60$\tabularnewline
\hline
{\bfseries High}&&&&&&&\tabularnewline
~~JFM&$0.03$&$0.43$&$1.65$&&$0.00$&$ 3.20$&$ 55.80$\tabularnewline
~~AMJ&$0.02$&$0.48$&$1.95$&&$0.10$&$ 5.40$&$ 74.90$\tabularnewline
~~JAS&$0.03$&$0.54$&$3.16$&&$0.10$&$ 7.23$&$333.40$\tabularnewline
~~OND&$0.02$&$0.43$&$2.43$&&$0.00$&$ 4.67$&$142.90$\tabularnewline
~~1977-1987&$0.02$&$0.57$&$1.92$&&$0.30$&$ 7.30$&$136.80$\tabularnewline
~~1987-1997&$0.02$&$0.54$&$2.43$&&$0.00$&$ 5.11$&$142.90$\tabularnewline
~~1997-2007&$0.02$&$0.56$&$3.16$&&$0.00$&$ 4.80$&$ 72.30$\tabularnewline
~~2007-2017&$0.03$&$0.33$&$1.80$&&$0.80$&$ 3.70$&$333.40$\tabularnewline
\hline
\end{tabular}\end{center}
\end{table}

The consistent decline in chlorophyll concentrations were opposite to
the observed trends in the number and types of restoration projects in
the watershed. As shown in \cref{fig:restyrs}, less projects were
observed early in the record, whereas a majority were completed after
the year 2000. Individual point source controls early in the record were
those that occurred in the historically polluted upper Hillsborough Bay
(Johansson \protect\hyperlink{ref-Johansson91}{1991}; Johansson and
Lewis III \protect\hyperlink{ref-Johansson92}{1992}). Prior to 1995,
only 11 water infrastructure projects (three non-point control, eight
point source controls) were found in the database, whereas 70 habitat
projects were observed (50 habitat establishment, 20 habitat
enhancement). From 1995 and on, nearly ten times as many restoration
projects were observed in the record (806 total) with notable increases
in the number of nonpoint source controls (245) and habitat protection
projects (45). For the entire record, 275 (31\% of total) habitat
enhancement, 259 (29\%) habitat establishment, 45 (5\%) habitat
protection, 248 (28\%) nonpoint source, and 60 (7\%) point source
control projects were observed.

\hypertarget{relationships-of-restoration-projects-with-water-quality}{%
\subsection{Relationships of restoration projects with water
quality}\label{relationships-of-restoration-projects-with-water-quality}}

A simple comparison of water quality measurements versus the cumulative
number of restoration projects over time showed a decrease in both total
nitrogen and chlorophyll with additional restoration. \Cref{fig:cumprj}
shows median water quality estimates across all monitoroing stations for
a given year against the total number of cumulative restoration projects
for the current and all preceding years. Significant relationships were
observed between water quality and number of projects for all project
types (\(\alpha\) = 0.05), although the strength of association varied.
Overall, stronger assocations with number of projects were observed for
total nitrogen and relatively weaker assocations were observed for
chlorophyll. Decreases in total nitrogen were most strongly associated
with water infrastructure projects for nonpoint source
(\(F=65.5, df = 1, 23, p < 0.005\)) and point source controls
(\(F=60.8, df = 1, 21, p < 0.005\)), as expected. Habitat protection
projects were also strongly associated with decreases in total nitrogen
(\(F=34.8, df = 1, 14, p < 0.005\)). For chlorophyll, the strongest
associations were observed with habitat establishment
(\(F=20.8, df = 1, 35, p< 0.005\)) and point source control
(\(F=13.7, df = 1, 22, p < 0.005\)) projects. A weak but marginally
significant association was observed between chlorophyll and habitat
protection projects (\(F=4.6, df = 1, 14, p = 0.049\)).

\begin{figure}
\centerline{\includegraphics[width = 0.95\textwidth]{figs/cumprj.pdf}}
\caption{Relationships between cumulative number of restoration projects over time and water quality observations in Tampa Bay. The plot shows median total nitrogen (mg/L) and chlorophyll ($\mu$g/L) across all monitoring stations for each year against the cumulative number of projects for all preceding years.  Points are sized and shaded by year to show the progression of water quality and number of projects over time.  Summary statistics are shown in the bottom left corner as the significance of the linear regression and R-squared value. \textit{p > 0.05 ns, p < 0.05 *, p < 0.005 **}}
\label{fig:cumprj}
\end{figure}

An obvious limitation of the above analysis is the confounding effect of
almost all restoration projects increasing through time. Although
significant assocations were observed with improvements in water quality
and increasing cumulative number of projects, the comparisons in
\cref{fig:cumprj} do not provide a means to distinguish effects of
different project types. For example, the separate effects of habitat
establishment and point source projects on chlorophyll reductions cannot
be separated through simple linear analyses because both increase over
time. An assocation of chlorophyll with one project type could b an
artifact of an assocation with another project type. Likewise, a weak
assocation (e.g., habitat protection and chlorophyll) does not provide
strong evidence that a particular project type is unimportant for water
quality improvements. Our formal approach that uses spatial-temporal
matching between restoration projects and water quality stations is
meant to overcome these challenges and results from the simple analysis
above provides a basis of comparison for how our approach could lead to
new insights.

Baywide estimates of the potential effects of restoration projects using
spatial-temporal matching differed depending on the year windows and
number of closest restoration projects that were matched to each water
quality station. The results are shown in \cref{fig:prjsig} where the
estimated associations of different projects types with chlorophyll at
individual stations are shown on the left maps and the aggregate
assocations across all stations for a given project type are shown in
the right plots. That is, station points in the maps correspond to an
estimate for the year window and closest project type selections for
each project type that were obtained through the steps in
\cref{fig:statex} and \cref{eq:wqdif}. Water quality stations outlined
in black show where the estimated change is significant (i.e., the
standard error lines in the botom plot of \cref{fig:statex} do not
include zero, green points where chlorophyll was lower, red points where
higher). The plots on the right are based on the distributions of the
estimated water quality changes for each station for the corresponding
project types in the maps on the left. The plots on the right also
include statistical summaries for 1) an analysis of variance (ANOVA)
F-test to compare the distribution of water quality changes between
project types, 2) individual t-tests for each projct type to evaluate
changes that were different from zero, and 3) a multiple comparison test
denoted by letters to identify which projects types had changes that
were different from each other.

For site-specific estimates of water quality changes, a slight trend of
longer time windows and more project matches with the number of
monitoring stations that had significant changes was observed (i.e.,
more black circles in the maps \cref{fig:prjsig}d, compared to
\cref{fig:prjsig}a). This was particularly true for habitat protection
projects where no significant associations were observed for the 5 year
window, 5 closest projects combination, but twelve stations had
significant assocations for the 10 year window, 10 closest projects
combination. A similar trend was observed for point source control
projects where more stations had more significant reductions in
chlorophyll with the 10 year window, 10 closest projects. The largest
number of stations (n = 13) with significant improvements in water
quality for nonpoint source projects was observed for the 5 year window,
10 closest projects. Associations of habitat enhancement and habitat
establishment projects with water quality stations were inconsistent,
with some sites showing an increase or decrease that varied by the year
window, closest project combinations. Spatial patterns among stations
regarding associations with different project types were not clear,
although point source controls were more commonly associated with
mid-bay stations (Middle Tampa Bay).

The estimated baywide effects for each project type showed that point
source controls were more strongly associated with reductions in
chlorophyll than the other project types (\cref{fig:prjsig}, right
plots). This assocation was particularly strong for the ten year window
combinations (\cref{fig:prjsig}c, d), where the results suggested an
overall baywide reduction in chlorophyll of aproximately 2 \(\mu\)g/L
(based on the median change across all sites, reduction of 2.7
\(\mu\)g/L for 10 years, 5 closest projects and 1.6 \(\mu\)g/L for 10
years, 10 closest projects).

Nonpoint controls were also significant but only when more projects were
considered. Habitat protection was also important at all combinations,
habitat establish only important with more projects and time, whereas
habitat enhancement had a negligible association no matter the
comparison parameters.

Segment results in \cref{tab:prjsigseg}.

\begin{figure}
\includegraphics[width=1\linewidth]{figs/prjsig} \caption{Associations of restoration projects with chlorophyll changes at all sites in Tampa Bay from 1974 to 2017.  Associations were evaluated based on different year windows (5, 10) since completion of restoration projects and number of closest restoration projects (5, 10) to each monitoring station (subfigures a-d).  The left plots show the estimated changes at each site (green decreasing, red increasing) for each restoration project type, with significant changes at a site outlined in black.  The right plots show the aggregated site changes for each project type.  Overall differences were evaluated by ANOVA F-tests (bottom left corner), whereas pairwise differences between project types were evaluated by t-tests with corrected p-values for multiple comparisons.  Chlorophyll changes by project types within each subfigure that are not significantly different share a letter and significance of the within-group mean relative to zero is also shown.}\label{fig:prjsig}
\end{figure}

\begin{table}[!tbp]
\caption{Associations of restoration projects with chlorophyll changes for different segments of Tampa Bay from 1974 to 2017.  Associations were evaluated based on different year windows (5, 10) since completion of restoration projects and number of closest restoration projects (5, 10) to each monitoring station within each segment. Overall differences in chlorophyll changes between restoration project types by segment and year/project number combinations were evaluated by ANOVA F-tests, whereas pairwise differences between project types were evaluated by t-tests with corrected p-values for multiple comparisons.  Chlorophyll changes by project types that are not significantly different share a letter and significance of the within-group mean relative to zero is also shown.\label{tab:prjsigseg}} 
\begin{center}
\begin{tabular}{llclllll}
\hline\hline
\multicolumn{1}{l}{\bfseries Bay segment}&\multicolumn{1}{c}{\bfseries }&\multicolumn{1}{c}{\bfseries }&\multicolumn{5}{c}{\bfseries Restoration projects}\tabularnewline
\cline{4-8}
\multicolumn{1}{l}{}&\multicolumn{1}{c}{ANOVA}&\multicolumn{1}{c}{}&\multicolumn{1}{c}{\thead{Habitat\\enhance}}&\multicolumn{1}{c}{\thead{Habitat\\establish}}&\multicolumn{1}{c}{\thead{Habitat\\protect}}&\multicolumn{1}{c}{\thead{Nonpoint\\control}}&\multicolumn{1}{c}{\thead{Point\\control}}\tabularnewline
\hline
{\bfseries 5 years, 5 projects}&&&&&&&\tabularnewline
~~HB&\textit{F = 1.09, ns}&&a, 0&a, 0&a, $\textless $ 0&a, 0&a, 0\tabularnewline
~~LTB&\textit{F = 17.37, **}&&a, 0&a, 0&a, 0&a, 0&b, $\textless $ 0\tabularnewline
~~MTB&\textit{F = 13.5, **}&&b, 0&ab, 0&a, $\textless $ 0&ab, $\textless $ 0&c, $\textless $ 0\tabularnewline
~~OTB&\textit{F = 1.5, ns}&&a, 0&a, 0&a, 0&a, 0&a, 0\tabularnewline
\hline
{\bfseries 5 years, 10 projects}&&&&&&&\tabularnewline
~~HB&\textit{F = 0.66, ns}&&a, 0&a, 0&a, 0&a, 0&a, 0\tabularnewline
~~LTB&\textit{F = 2.64, ns}&&ab, $\textless $ 0&ab, $\textless $ 0&ab, $\textless $ 0&a, 0&b, $\textless $ 0\tabularnewline
~~MTB&\textit{F = 18.75, **}&&a, 0&a, 0&a, $\textless $ 0&a, $\textless $ 0&b, $\textless $ 0\tabularnewline
~~OTB&\textit{F = 3.11, *}&&ab, 0&a, 0&ab, 0&b, $\textless $ 0&a, 0\tabularnewline
\hline
{\bfseries 10 years, 5 projects}&&&&&&&\tabularnewline
~~HB&\textit{F = 2.9, *}&&ab, 0&ab, 0&ab, $\textless $ 0&a, 0&b, $\textless $ 0\tabularnewline
~~LTB&\textit{F = 6.13, **}&&a, 0&a, 0&a, 0&ab, 0&b, $\textless $ 0\tabularnewline
~~MTB&\textit{F = 14.11, **}&&a, 0&a, 0&a, 0&a, 0&b, $\textless $ 0\tabularnewline
~~OTB&\textit{F = 3.15, *}&&b, 0&ab, 0&ab, 0&a, $\textless $ 0&a, $\textless $ 0\tabularnewline
\hline
{\bfseries 10 years, 10 projects}&&&&&&&\tabularnewline
~~HB&\textit{F = 2.42, ns}&&a, 0&a, 0&a, 0&a, 0&a, $\textless $ 0\tabularnewline
~~LTB&\textit{F = 1.78, ns}&&a, 0&a, $\textless $ 0&a, $\textless $ 0&a, 0&a, $\textless $ 0\tabularnewline
~~MTB&\textit{F = 11.79, **}&&a, 0&a, $\textless $ 0&a, 0&a, 0&b, $\textless $ 0\tabularnewline
~~OTB&\textit{F = 2.35, ns}&&b, 0&ab, $\textless $ 0&ab, 0&a, $\textless $ 0&ab, $\textless $ 0\tabularnewline
\hline
\end{tabular}\end{center}
\footnotesize HB: Hillsborough Bay, LTB: Lower Tampa Bay, MTB: Middle Tampa Bay, OTB: Old Tampa Bay. \textit{p > 0.05 ns, p < 0.05 *, p < 0.005 **}\end{table}

\hypertarget{discussion}{%
\section{Discussion}\label{discussion}}

A long-term record of restoration activities and water quality data in
Tampa Bay provided the foundation to develop a novel decision support
tool for coastal restoration practitioners and managers. This new tool
provides an improved process to understand the expected water quality
improvements that could result from future restoration activities
contingent upon the level of cumulative investments made toward
divergent activities and the required monitoring to understand
downstream water quality benefits at local to watershed-wide scales.
This tool has broad application and extension within the Gulf Coast
restoration and management community. However, several reservations on
its application were discovered.

\begin{itemize}
\item
  2004-2017, nonpoint source and habitat protection appears to have the
  largest effect on chlorophyll-a levels. This effect is not clear until
  after 5 years of monitoring and with the evaluation of multiple
  projects. When fewer restoration activities were taking place in the
  Bay (i.e.~94-04) a greater monitoring time frame was necessary to
  identify the benefits of restoration activities (\textgreater{}5
  years).
\item
  From 1974 - 1994, chlorophyll changes were most distinct in relation
  to water infrastructure projects, particularly for low salinity
  regions in the bay. The effect was more apparent with increasing time
  from the completion of a project and with evaluation of multiple
  projects. Habitat projects did not have a noticeable effect although,
  these were limited to later in the time period.
\item
  Our analysis of the different year window, number of matches provided
  an approach to separately evaluate the effect of temporal signals and
  spatial signals on water quality improvements. Explain\ldots{}
\item
  What can we really say with this analysis? Above this text, describe
  in bullet points the key conclusions, but here we can describe what
  they really mean, e.g.~why did some sites have reduction in water
  quality with projects? These are noisy relationships\ldots{}. Our
  approach provides a tradeoff between the noise and ability to
  disentangle cumulative effects but it's messy. The figures on the
  right in figure 7 provide one approach to identify the signal to wash
  out what may be errors.
\item
  It is inherently difficult to determine any downstream water quality
  benefits from enhancement no matter how many sites or years.
  Enhancement is primarily activities such as invasive species removal
  which is not done with the primary goal of improving water quality.
\item
  How can this be used to guide coastal wq management in the Gulf?

  \begin{itemize}
  \tightlist
  \item
    Demonstrate the benefit of Long Term monitoring
  \end{itemize}
\item
  How can this be used to inform or prioritize restoration activities in
  the Gulf?
\item
  What are the limitations of our analysis?
\item
  How can the analysis be applied in other locations?
\end{itemize}

\hypertarget{limitations-of-analysis}{%
\subsection{Limitations of analysis}\label{limitations-of-analysis}}

\begin{itemize}
\item
  More restoration projects later in the time series could be from
  inconsistent data collection early in the time series
\item
  Associative by design - does not empirically link cause and effect,
  actual relationships are very complex
\end{itemize}

\hypertarget{references}{%
\section*{References}\label{references}}
\addcontentsline{toc}{section}{References}

\hypertarget{refs}{}
\leavevmode\hypertarget{ref-Alfredo17}{}%
Alfredo, K. A., and T. A. Russo. 2017. ``Urban, Agricultural, and
Environmental Protection Practices for Sustainable Water Quality.''
\emph{WIREs Water} 4 (5):e1229. \url{https://doi.org/10.1002/wat2.1229}.

\leavevmode\hypertarget{ref-Bayraktarov16}{}%
Bayraktarov, Elisa, Megan I. Saunders, Sabah Abdullah, Morena Mills,
Jutta Beher, Hugh P. Possingham, Peter J. Mumby, and Catherine E.
Lovelock. 2016. ``The cost and feasibility of marine coastal
restoration.'' \emph{Ecological Applications} 26 (4):1055--74.
\url{https://doi.org/10.1890/15-1077}.

\leavevmode\hypertarget{ref-Beck15}{}%
Beck, M. W., and J. D. Hagy III. 2015. ``Adaptation of a Weighted
Regression Approach to Evaluate Water Quality Trends in an Estuary.''
\emph{Environmental Modelling and Assessment} 20 (6):637--55.
\url{https://doi.org/10.1007/s10666-015-9452-8}.

\leavevmode\hypertarget{ref-Beck17c}{}%
Beck, M. W., J. D. Hagy III, and C. Le. 2017. ``Quantifying Seagrass
Light Requirements Using an Algorithm to Spatially Resolve Depth of
Colonization.'' \emph{Estuaries and Coasts}, 1--17.

\leavevmode\hypertarget{ref-Beyer16}{}%
Beyer, Jonny, Hilde C. Trannum, Torgeir Bakke, Peter V. Hodson, and
Tracy K. Collier. 2016. ``Environmental effects of the Deepwater Horizon
oil spill: A review.'' \emph{Marine Pollution Bulletin} 110 (1):28--51.
\url{https://doi.org/10.1016/j.marpolbul.2016.06.027}.

\leavevmode\hypertarget{ref-Borja16}{}%
Borja, Ángel, Guillem Chust, José G. Rodríguez, Juan Bald,
M\textordfeminine Jesús Belzunce-Segarra, Javier Franco, Joxe Mikel
Garmendia, et al. 2016. ```The past is the future of the present':
Learning from long-time series of marine monitoring.'' \emph{Science of
the Total Environment} 566-567. Elsevier B.V.:698--711.
\url{https://doi.org/10.1016/j.scitotenv.2016.05.111}.

\leavevmode\hypertarget{ref-Diefenderfer16}{}%
Diefenderfer, Heida L., Gary E. Johnson, Ronald M. Thom, Kate E. Buenau,
Laurie A. Weitkamp, Christa M. Woodley, Amy B. Borde, and Roy K. Kropp.
2016. ``Evidence-Based Evaluation of the Cumulative Effects of Ecosystem
Restoration.'' \emph{Ecosphere} 7 (3):e01242.
\url{http://dx.doi.org/10.1002/ecs2.1242}.

\leavevmode\hypertarget{ref-Enwright16}{}%
Enwright, Nicholas M., Kereen T. Griffith, and Michael J. Osland. 2016.
``Barriers to and opportunities for landward migration of coastal
wetlands with sea-level rise.'' \emph{Frontiers in Ecology and the
Environment} 14 (6):307--16. \url{https://doi.org/10.1002/fee.1282}.

\leavevmode\hypertarget{ref-Greening06b}{}%
Greening, Holly, Peter Doering, and Catherine Corbett. 2006. ``Hurricane
impacts on coastal ecosystems.'' \emph{Estuaries and Coasts} 29
(6):877--79. \url{https://doi.org/10.1007/BF02798646}.

\leavevmode\hypertarget{ref-Greening2014}{}%
Greening, H S, A Janicki, E T Sherwood, R Pribble, and J O R Johansson.
2014. ``Ecosystem responses to long-term nutrient management in an urban
estuary: Tampa Bay, Florida, USA.'' \emph{Estuarine, Coastal and Shelf
Science} 151 (December):A1--A16.
\url{https://doi.org/10.1016/j.ecss.2014.10.003}.

\leavevmode\hypertarget{ref-Hobbs01}{}%
Hobbs, R J, and J A Harris. 2001. ``Restoration Ecology: Repairing the
Earth's Ecosystems in the New Millennium.'' \emph{Restoration Ecology} 9
(2):239--46. \url{https://doi.org/10.1046/j.1526-100x.2001.009002239.x}.

\leavevmode\hypertarget{ref-Janicki99}{}%
Janicki, A., D. Wade, and J. R. Pribble. 1999. ``Development of a
Process to Track the Status of Chlorophyll and Light Attenuation to
Support Seagrass Restoration Goals in Tampa Bay.'' 04-00. St.
Petersburg, Florida: Tampa Bay National Estuary Program.

\leavevmode\hypertarget{ref-Johansson91}{}%
Johansson, J. O. R. 1991. ``Long-Term Trends of Nitrogen Loading, Water
Quality and Biological Indicators in Hillsborough Bay, Florida.'' Edited
by S. F. Treat and P. A. Clark. Tampa, Florida, USA: Tampa Bay Area
Study Group Project at Scholar Commons, 157--76.

\leavevmode\hypertarget{ref-Johansson92}{}%
Johansson, J. O. R., and R. R. Lewis III. 1992. ``Recent Improvements in
Water Quality and Biological Indicators in Hillsborough Bay, a Highly
Impacted Subdivision of Tampa Bay, Florida, USA.'' \emph{Marine Coastal
Eutrophication} Proceedings of an International Conference, Bologna,
Italy, 21-24 March 1990:1199--1215.

\leavevmode\hypertarget{ref-Morrison06}{}%
Morrison, G., E. T. Sherwood, R. Boler, and J. Barron. 2006.
``Variations in Water Clarity and Chlorophyll\emph{a} in Tampa Bay,
Florida, in Response to Annual Rainfall, 1985-2004.'' \emph{Estuaries
and Coasts} 29 (6):926--31.

\leavevmode\hypertarget{ref-Poe05}{}%
Poe, A., K. Hackett, S. Janicki, R. Pribble, and A. Janicki. 2005.
``Estimates of Total Nitrogen, Total Phosphorus, Total Suspended Solids,
and Biochemical Oxygen Demand Loadings to Tampa Bay, Florida:
1999-2003.'' \#02-05. St. Petersburg, Florida, USA: Tampa Bay Estuary
Program.

\leavevmode\hypertarget{ref-Rabotyagov14}{}%
Rabotyagov, S. S., C. L. Kling, P. W. Gassman, N. N. Rabalais, and R. E.
Turner. 2014. ``The Economics of Dead Zones: Causes, Impacts, Policy
Challenges, and a Model of the Gulf of Mexico Hypoxic Zone.''
\emph{Review of Environmental Economics and Policy} 8 (1):58--79.
\url{https://doi.org/10.1093/reep/ret024}.

\leavevmode\hypertarget{ref-RDCT18}{}%
RDCT (R Development Core Team). 2018. ``R: A language and environment
for statistical computing, v3.5.1. R Foundation for Statistical
Computing, Vienna, Austria.''

\leavevmode\hypertarget{ref-Schiff16}{}%
Schiff, K., P.R. Trowbridge, E.T. Sherwood, P. Tango, and R.A. Batiuk.
2016. ``Regional Monitoring Programs in the United States: Synthesis of
Four Case Studies from Pacific, Atlantic, and Gulf Coasts.''
\emph{Regional Studies in Marine Science} 4:A1--A7.
\url{https://doi.org/10.1016/j.rsma.2015.11.007}.

\leavevmode\hypertarget{ref-Sherwood16}{}%
Sherwood, E. T., H. S. Greening, A. J. Janicki, and D. J. Karlen. 2016.
``Tampa Bay Estuary: Monitoring Long-Term Recovery Through Regional
Partnerships.'' \emph{Regional Studies in Marine Science} 4:1--11.
\url{https://doi.org/10.1016/j.rsma.2015.05.005}.

\leavevmode\hypertarget{ref-Sherwood17}{}%
Sherwood, E. T., H. S. Greening, J. O. R. Johansson, K. Kaufman, and G.
Raulerson. 2017. ``Tampa Bay (Florida, USA): Documenting Seagrass
Recovery Since the 1980s and Reviewing the Benefits.''
\emph{Southeastern Geographer} 57 (3):294--319.

\leavevmode\hypertarget{ref-TBEP17}{}%
TBEP (Tampa Bay Estuary Program). 2017. ``Tampa Bay Water Atlas.''

\leavevmode\hypertarget{ref-Wang99}{}%
Wang, P. F., J. Martin, and G. Morrison. 1999. ``Water Quality and
Eutrophication in Tampa Bay, Florida.'' \emph{Estuarine, Coastal and
Shelf Science} 49 (1):1--20.


\end{document}
